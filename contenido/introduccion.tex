\chapter*{Introducción}
\addcontentsline{toc}{chapter}{Introducción}

Las matemáticas y la música han mantenido, desde el origen de la civilización, una relación profunda que transciende más allá del arte y de la ciencia. Ambas disciplinas comparten una esencia común: la búsqueda de estructuras, patrones y proporciones que permiten comprender mejor la realidad y dotarla de sentido.
Si las matemáticas se han concebido tradicionalmente como el lenguaje del rigor y la abstracción para establecer una base común a todos los sucesos científicos, la música, por su parte, ha sido entendida como una manifestación artística ligada a la sensibilidad y la emoción. Sin embargo, el contacto entre ambas ha demostrado que son capaces de confluir en un terreno fértil, donde las proporciones numéricas se convierten en armonías y donde las estructuras formales encuentran una expresión sonora. Tanto es así que en ocasiones resulta complejo discernir entre aquello que 
determina cuando algo es arte y cuando es ciencia.

El trabajo que se ha realizado se ha dividido en X capítulos, tal, pascual, manel y pel, que explican algunos de los fenómenos más característicos.

\section*{Historia}
\addcontentsline{toc}{section}{Historia}

La relación entre matemáticas y música hunde sus raíces en la Antigüedad Clásica. La propia etimología de los términos refleja su proximidad conceptual. Matemáticas proviene del griego {\it máthēma} (“conocimiento” o “aprendizaje”), mientras que música procede de {\it mousikê}, expresión que aludía al “arte de las Musas”, un conjunto de prácticas que incluían la poesía, el canto y la danza, concebidas como vehículos de transmisión cultural y de formación intelectual.

En la Grecia Clásica, el conocimiento se entendía como un todo integrado, en el que disciplinas hoy diferenciadas se concebían como manifestaciones de un mismo orden racional. La música ocupaba un lugar central, pues no solo se valoraba su dimensión estética, sino también su capacidad de expresar relaciones armónicas que se interpretaban como reflejo del Cosmos. En este contexto se inscribe la Escuela Pitagórica, movimiento filosófico-religioso surgido a principios del siglo VI a.C., que trataba de dotar a los elementos del mundo de un transfondo numérico. Este grupo de pensadores, liderados por \textit{Pitágoras de Samos}, logró establecer relaciones matemáticas entre las consonancias fundamentales, que podían describirse mediante proporciones numéricas simples: 2/1 para la octava, 3/2 para la quinta, 4/3 para la cuarta... Este hallazgo, acompañado de la manipulación de instrumentos musicales rudimentarios como el monocordio, dio lugar a la idea de la \textit{Armonía de las Esferas}, según la cual el universo entero podía comprenderse como una gran composición musical regida por proporciones matemáticas.
\begin{figure}
    \centering
    \includegraphics[width=0.5\linewidth]{Plantilla-LaTeX-TFG/images/Gaffurio_Pythagoras-743x1024.png}
    \caption{Gaffurio, \textit{Theorica musice}, 1492}
    \label{fig:placeholder}
\end{figure}

En el Medievo temprano, autores como \textit{Boecio} sistematizaron el saber heredado en tratados como el \textit{De institutione musica}, que influyeron decisivamente en la enseñanza medieval. \\
En este periodo, el conocimiento de los hombres libres se congregaba en las sietes artes liberales, en contraposición con las artes serviles propias de los esclavos. Las siete artes liberales se dividían entre el \textit{Trivium} (retórica, gramática y dialéctica) y el \textit{Quadrivium}, que incorporaba las disciplinas ''matemáticas'' (aritmética, geometría, astronomía y música.) De esta manera se acabó de concebir la música como una ciencia matemática. Los teóricos musicales medievales escribieron tratados que exploraban las bases matemáticas de la música, continuando la tradición pitagórica y aristotélica y enfatizando la importancia de las proporciones para construir escalas y comprender la armonía.

Con la llegada de la polifonía durante el Barroco, compositores como \textit{J.S. Bach} produjeron repertorio que reveló el diseño \textit{quasi} algorítmico y simétrico de las composiciones musicales. Fue el contrapunto, el arte de combinar varias líneas melódicas independientes pero armónicamente relacionadas, el que demostró la intrínseca lógica musical. En la \textit{fuga barroca}, por ejemplo, un tema musical se presenta y luego es imitado por otras voces en diferentes tonalidades y momentos, siguiendo reglas estrictas de entrada, superposición y desarrollo temático que recuerdan a permutaciones y transformaciones matemáticas. 

Más tarde, el Clasicismo, con figuras como \textit{Mozart} y \textit{Haydn}, elevó la forma \textit{sonata} a su máxima expresión. Esta estructura musical tripartita (exposición, desarrollo y recapitulación) presenta un equilibrio y una proporción clásicos, donde los temas y las tonalidades se presentan, exploran y resuelven de manera sistemática y predecible, aunque siempre con un genio creativo que trasciende la mera fórmula. La búsqueda de claridad, orden y balance en estas formas resonaba directamente con los ideales de la Ilustración, donde la razón y la lógica matemática eran pilares fundamentales del conocimiento.

A partir del siglo XX, la relación entre música y matemáticas se hizo aún más explícita y consciente, dando lugar a nuevas corrientes compositivas y herramientas de análisis. El desarrollo del dodecafonismo por \textit{Arnold Schoenberg} fue un intento deliberado de aplicar principios matemáticos rigurosos a la composición. En este sistema, las doce notas de la escala cromática se organizan en una ``serie'' específica, que luego se manipula mediante transformaciones como la inversión y la retrogradación. Cada nota de la serie debe sonar antes de que cualquiera pueda repetirse, asegurando una distribución igualitaria de todas ellas y evitando la tonalidad tradicional. Este enfoque, eminentemente matemático, buscaba una nueva lógica estructural para la música atonal.

Además, la proliferación de la tecnología y los ordenadores en la segunda mitad del siglo XX abrió las puertas a la música algorítmica y la composición asistida por ordenador. Compositores como \textit{Iannis Xenakis} utilizaron la teoría de probabilidades, los procesos estocásticos y la síntesis granular para crear obras donde la estructura musical emergía de complejos modelos matemáticos. La música se generaba a partir de reglas predefinidas, explorando la relación entre el orden y el caos, y demostrando cómo las matemáticas pueden ser una herramienta no solo para analizar, sino también para crear nuevas sonoridades y estructuras musicales.

Hoy en día, la interconexión entre ambas disciplinas sigue evolucionando. La física acústica, la psicoacústica y la informática musical utilizan modelos matemáticos avanzados para comprender cómo se produce, propaga y percibe el sonido. Desde la síntesis de sonido y el procesamiento de señales digitales hasta el análisis de grandes conjuntos de datos musicales, las matemáticas son indispensables. En esencia, la música, en su expresión más abstracta y universal, revela patrones, simetrías y relaciones que son profundamente matemáticas, y que continúan siendo una fuente inagotable de inspiración tanto para músicos como para matemáticos.



\section*{Motivación}
\addcontentsline{toc}{section}{Motivación}


{\Large Hacerlo impersonal. Preguntas retóricas y así.}
Personal: La coincidencia de mi formación en matemáticas e informática con mi práctica instrumental (trompa) me ha mostrado repetidamente cómo herramientas matemáticas permiten describir y resolver problemas musicales concretos: afinación, análisis espectral del timbre, modelado de articulaciones, generación algorítmica de motivos, etc. Este proyecto busca justificar y profundizar ese puente, produciendo tanto conocimiento teórico como herramientas concretas útiles para la práctica y la enseñanza musical.

Académica / científica: Desde la modelización de temperamentos y sistemas de afinación hasta la aplicación de transformadas (Fourier, wavelets) para el análisis de timbre, las matemáticas aportan lenguajes y técnicas que amplían la comprensión musical y la capacidad de innovación (composición algorítmica, síntesis, análisis computacional). Además, la integración con la informática permite implementar, visualizar y experimentar con modelos que de otro modo serían meramente teóricos.


\section*{Objetivos del trabajo}
\addcontentsline{toc}{section}{Objetivos}

Este trabajo pretende desarrollar un estudio generalizado de la relación ``simbiótica'' que mantienen las matemáticas y la música. Para lograrlo, se ha hecho un recorrido por los principales elementos que definen la música, abordándolos desde una visión matemática y tratando de formalizarlos a través de conceptos propios de las ciencias exactas. 

Particularmente, se ha llevado a cabo una revisión del origen histórico de la relación entre ambas disciplinas que ha permitido conocer y aplicar herramientas algebraicas a la descripción de estructuras musicales, así como implementar y desarrollar ejemplos tanto computacionales como gráficos capaces de comparar sistemas de afinación, escalas musicales, progresiones de acordes...

En resumen,


{\Large Ya le daré forma más adelante, por ahora ''poc a poc i bona lletra.''}
Analizar la interacción entre las matemáticas y la teoría musical, integrando el estudio histórico, la formalización teórica y la experimentación computacional.

Objetivos específicos

Revisar el origen histórico de la relación entre música y matemáticas, desde la tradición pitagórica hasta su consolidación en el quadrivium medieval.

Formalizar matemáticamente conceptos musicales como intervalos, escalas, sistemas de afinación y consonancia.

Aplicar herramientas algebraicas y combinatorias (teoría de conjuntos, teoría de grupos) a la descripción de estructuras musicales.

Implementar análisis espectrales para el estudio del timbre, con especial atención a instrumentos de viento-metal.

Desarrollar ejemplos computacionales que permitan visualizar y experimentar con modelos teóricos, utilizando lenguajes de programación como Python.

Comparar diferentes sistemas de afinación y temperamento desde un enfoque matemático y perceptivo.

Explorar las implicaciones pedagógicas y musicológicas de la integración entre matemáticas y música.
